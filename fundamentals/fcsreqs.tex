
\subsubsection{Open Access and Development Practices}

The proprietary concerns of research institutions and security constraints of
data within fuel cycle simulators often restrict access. Use of a simulator is
therefore often limited to its institution of origin, necessitating effort
duplication at other institutions and thereby squandering broader human
resources. License agreements and institutional approval are required for most
current simulators (e.g., \gls{COSI}6, \gls{DANESS}, \gls{DESAE}, EVOLCODE,
FAMILY21, \gls{NFCSim})\cite{juchau_modeling_2010}, including ORION,
and \gls{VISION}.  Even when, as in the case of the MIT \gls{CAFCA} software,
the source code is unrestricted, the platform on which it relies is often
restricted or costly.  However, \Cyclus provides fully free and open access to
all users and developers, foreign and domestic.

Moreover, both technical and institutional aspects of the software development
practices employed by the \Cyclus community faciltate collaboration.
Technically, \Cyclus employs a set of tools commonly used collaborative
software development that reduce the effort required to comment on, test and
ultimately merge individual contributions into the main development path.
For many of the simulation platforms adopted by previous simulators, there were
technical obstacles that impeded this kind of collaboration.
Institutionally, \Cyclus invites all participants to propose, discuss and
provide input to the final decision making for all important changes.

\subsubsection{Modularity and Extensibility}

Modularity is a key enabler of extending the scope of fuel cycle analysis
within the \Cyclus framework.  Changes that are required to improve the
fidelity of modeling a paticular agent, or to introduce entirely new agents,
are narrowly confined and place no new requirements on the \Cyclus kernel.
Furthermore, there are very few assumptions or heuristics that would otherwise
restrict the algorithmic complexity that can be used to model the behavior of
such agents.

For example, most current simulators describe a finite set of acceptable cycle
constructions (once through, single-pass, multi-pass). That limits the
capability to create novel material flows and economic scenarios. The \Cyclus
simulation logic relies on a market paradigm, parameterized by the user, which
flexibly simulates dynamic responses to pricing, availability, and other
institutional preferences.

This minimal set of mutual dependencies between the kernel and the agents is
expressed through the \gls{DRE} that provides a level of flexibility that does
not exist in other fuel cycle simulators.  It creates the potential for novel
agent archetypes to interact with existing archetypes as they enter and leave
the simulation over time and seek to trade materials whose specific
composition may not be known \textit{a priori}.

\subsubsection{Discrete Facilities and Materials}

Many fuel cycle phenomena have aggregate system-level effects which can only be
captured by discrete material tracking \cite{huff_next_2010}.  \Cyclus
tracks materials as discrete objects. Some current fuel cycle simulation tools
such as \gls{COSI}
\cite{mccarthy_benchmark_2012,grasso_nea-wpfc/fcts_2009,guerin_benchmark_2009},
\gls{DESAE}
\cite{andrianova_desae_2008}, FAMILY21\cite{mccarthy_benchmark_2012},
\gls{GENIUS} version 1, \gls{GENIUS} version 2, and \gls{NFCSim} also possess the ability to
model discrete materials.

Similarly, the ability to model disruptions (i.e. facility shutdowns due to
insufficient feed material or insufficient processing and storage capacity) is
most readily captured by software capable of tracking the operations status of
discrete facilities \cite{huff_next_2010}.  Fleet-based models (i.e.
\gls{VISION}) are unable to capture this gracefully, perhaps representing it as
a reduction in the capacity of the whole fleet.  All of the software
capable of discrete materials have a notion of discrete facilities, however not
all handle disruption in the same manner. \gls{DESAE}, for example, does not
allow shutdown due to insufficient feedstock. In the event
of insufficient fissile material during reprocessing, \gls{DESAE} borrows
material from storage, leaving a negative value \cite{mccarthy_benchmark_2012}.
The \Cyclus framework does not dictate such heuristics. Rather, it provides a
flexible framework on which either method is possible.
