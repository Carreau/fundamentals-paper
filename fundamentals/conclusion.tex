

\section{Conclusions}
% summarize the conclusions of the paper as a list of short phrases or
% sentences. Do not repeat the results section unless special emphasis is
% needed. The conclusion is just that, not a summary. It should add a new,
% higher level of analysis and should explicitly indicate the significance of
% the work.
% What does it all mean?
% What hypotheses were proven or disproven?
% What was learned?
% Why does it make a difference?

%Modularity encourages ecosystem development
%Enables cross-institutional collaboration
%Open paradigm allows community oversight, testing, verification
%Enables apples to apples comparisons

The \Cyclus nuclear fuel cycle framework presents a more generic and flexible
alternative to existing fuel cycle simulators. Where previous nuclear fuel
cycle simulators have had limited distribution, constrained simulation
capabilities, and restricted customizability, \Cyclus emphasizes an open
strategy for access and development.  This open strategy not only improves
accessibility, but also enables transparency and community oversight.

The object-oriented \gls{ABM} simulation paradigm ensures more generic
simulation capability. It allows \Cyclus to address common analyses in a more
flexible fashion and enables analyses that are impossible with system dynamics
simulators.

Similarly the fidelity-agnostic modular architecture facilitates simulations
at every level of detail. Simulations relying on arbitrarily complex isotopic
compositions are possible in \Cyclus, as are simulations not employing any
physics at all. Indeed, agents of such varying fidelity can even exist in the
same simulation. Researchers no longer need to reinvent the wheel in order to
model a simulation focused on the aspects of the fuel cycle
relevant to their research.

Furthermore, when the capabilities within \Cyclus, \Cycamore, and the rest of
the ecosystem are insufficient, adding custom functionality is simplified by a
modular, plug-in architecture. A clean, modern \gls{API} simplifies
customization and independent archetype development so that researchers can
create models within their domain of expertise without modifying the core
simulation kernel. Throughout the \Cyclus
infrastructure, architecture choices have sought to enable cross-institutional
collaboration and sustainable, community-driven development. The ecosystem
of capabilities, already growing, may someday reflect the full diversity of use
cases in the nuclear fuel cycle simulation domain.


