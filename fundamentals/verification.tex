\section{Quality Assurance}
% Organize this section according to major topics
% give each topic a section heading in boldface.
% try to cover the major common points :
%
% problem design
% methods of measurement
% supporting models
% supporting data
% simulations run
% results

% Just write the section headings for each part and indicate what goes in that
% section with words :
%
% heading
% figures (with captions)
% schematics (with captions and footnotes)
% equations
% tables

% What does it mean?
% What did I actually test?
% What were the results?
% Did the work yield a new method?
% Did the work yield new knowledge?
% What measurements did I make?
% How were these measurements characterized?
% What methods were used?
% What were the results?
% How were the measurements made and characterized?

Unfortunately, we can't wind forward the clock to compare our simulations to 
the futures that we are attempting to model.

However, Cyclus does implement a number of strategies for verification and 
validation of code.

\subsection{Coding Practices}
% What have we done to ensure robustness ?
% Examples from Material class, Transactions, simulation IDs

Some routines are implemented specifically to ensure robustness.

\subsection{Testing Framework}
% Peer-review system (secretly a note on open source?)

We have reasonable test coverage and a code review system.

\subsubsection{Unit Tests}

Testing individual units of 'work'.

\subsubsection{Regression Tests}

Testing identical output for each pull request.

\subsubsection{Integration Tests} 

Testing aggregate answers for each pull request. Specifically, previous
simulations used in publications are integration tested to confirm their output
is maintained.

