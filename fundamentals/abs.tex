As nuclear power expands both domestically and internationally, nuclear fuel cycle
performance analysis on technical, economic, political, and environmental axes increases
in importance. Appropriate nuclear fuel cycle analysis requires calculation of myriad physical, nuclear,
chemical, industrial, and political phenomena. Nuclear fuel cycle simulators
therefore couple complex systems such as nuclear process physics,
facility deployment, and material routing. However, most nuclear fuel cycle
simulators have not incorporated modern insights in simulation science and
software architecture that can enable more efficient, accurate, robust, and
validated analysis.

An alternative to these approaches is a dynamic, agent-based simulator with an
open platform and an ecosystem of research-driven capabilities.  Modern
insight indicates that dynamic nuclear fuel cycle analysis more realistically
supports a broader range of simulation goals than static analysis
\cite{piet_dynamic_2011}. Similarly, experience in the broader field of systems
analysis indicates that agent-based modeling enables more flexible simulation
control, without loss of generality \cite{macal_agent-based_2010}. Finally, openness
allows cross-institutional collaboration, increases software robustness, and
enables the cultivation of an ecosystem of calculation options
\cite{softwarecarpentryresource}.  The fundamental concepts of the \Cyclus
nuclear fuel cycle simulator capture these modern insights so that challenges
in nuclear fuel cycle analysis can be better addressed. A summary of the
motivating use cases, an overview of the software architecture, and several
representative solutions built on the framework are presented.


