\section{Examples}
% Organize this section according to major topics
% give each topic a section heading in boldface.
% try to cover the major common points :
%
% problem design
% methods of measurement
% supporting models
% supporting data
% simulations run
% results

% Just write the section headings for each part and indicate what goes in that
% section with words :
%
% heading
% figures (with captions)
% schematics (with captions and footnotes)
% equations
% tables

% What does it mean?
% What did I actually test?
% What were the results?
% Did the work yield a new method?
% Did the work yield new knowledge?
% What measurements did I make?
% How were these measurements characterized?
% What methods were used?
% What were the results?
% How were the measurements made and characterized?

\subsection{Ecosystem}
% Cycamore library
% Cyder?

The archetypes created by user-developers and the archetypes provided in the 
\Cycamore \cite{carlsen_cycamore_2014} repository of additional modules 
together form an `ecosystem' of capabilities. Since the long term vision for 
the \Cyclus framework includes an ever-expanding ecosystem of both general and 
specialized capability extensions within that ecosytem. The early growth and 
cross-institutional contributions to this ecosystem demonstrates a significant 
acheivement by the \Cyclus framework. 

\subsubsection{The Cyclus Additional Modules Repository}

\Cycamore, the \Cyclus additional modules repository, provides a fundamental 
set of agent archetypes for basic simulation functionality within \Cyclus.
Since the \Cyclus framework relies on external archetypes to represent the 
agents within a simulation, 
\Cycamore provides the basic archetypes a new user needs to get started running 
simple simulations. 
These archetypes support a minimal set of fuel cycle simulation goals and 
provide, by example, a guide to new developers who would seek to contribute 
their own archetypes outside of \Cycamore.

\subsubsection{External Modules}
External modules have been 
\cite{cyder,separations,streamblender,mktdriveninst,commodconverter} and are 
being \cite{britelite,utk} developed for contribution to the cyclus ecosystem 
of models. 



\subsection{Simulations}
%FCO example (will be done soon.. publishable?)
%Inpro example (is this still running or did we deprecate it with 1.0?)

% MJG - INPRO should be tried again. @rwcarlsen has been running lots of
% simulations with the batch reactor, and its the second generation of my
% reactor models. We could easily use the same demand schedule and enrichment
% facility parameters and run it again.

Simulations have been run in the past and more are being run. FCO is an 
example. 
