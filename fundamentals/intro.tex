% Introduction :

\section{Introduction}

% Why did we do the work?
% What were the central motivations and hypotheses?
% The objectives of the work?
% What will this work show the reader?
% Why is it important?
% What should the reader watch for in the paper?
% What are the interesting high points?
% What strategy did we use?
% What should the reader expect as conclusion?


As nuclear power expands, technical, economic, political, and environmental analyses of nuclear fuel
cycles by simulators increases in importance. The merit of advanced nuclear technologies and the fuel cycles they comprise is shaped by myriad physical, nuclear, chemical, industrial, and political factors. Nuclear fuel cycle simulators must therefore couple complex models of
nuclear process physics, facility deployment, and material routing.
However, current fuel cycle simulators rely on closed platforms and inflexible
architectures which exhibit three main failure modes. First, they discourage
targeted contribution and collaboration among experts. Next, they hobble
efforts to directly compare modeling methodologies. Finally, they
over-specialize, rendering most tools applicable to only a subset of
desired simulation fidelities, scales, and applications.
The \Cyclus nuclear fuel cycle simulator framework and
its modeling ecosystem incorporate modern insights from
simulation science and software architecture to solve these problems.
These modern methods simultaneously enable more efficient, accurate,
robust, and validated analysis.

The \Cyclus next-generation fuel cycle simulator is the result of design
choices made to:
\begin{itemize}
\item support access to the tool by fuel cycle analysts and other users,
\item encourage developer extensions,
\item enable plug-and-play comparison of modeling methodologies,
\item and address a range of analysis types, levels of detail, and analyst sophistication.
\end{itemize}
Historically, nuclear fuel cycle simulators have calculated
fuel cycle mass balances and performance metrics derived from them using software
ranging from ad hoc spreadsheet-driven flow calculators to highly specialized
system dynamics modeling platforms. To date, current tools are either static, fleet-based,
privately distributed, or all of the above. Each of these qualities presents a
challenge to modeling fidelity, generality, efficiency, robustness, and
scientific transparency.

\Cyclus, therefore, is a dynamic, agent-based model, which employs a modular
architecture, an open development process, discrete agents, discrete time, and
arbitrarily detailed isotopic resolution of materials. Dynamic nuclear fuel
cycle analysis more realistically supports a range of simulation goals than
static analysis
\cite{piet_dynamic_2011}. Similarly, experience in the broader field of systems
analysis indicates that agent-based modeling enables more flexible simulation
control, without loss of generality \cite{macal_agent-based_2010}. Finally, openness
allows cross-institutional collaboration, increases code robustness \cite{cohen_modern_2010}, and
cultivates an ecosystem of modeling and calculation options.  Comprising flexible, interchangeable models of fuel cycle components with varying scope, depth and fidelity, this ecosystem allows users and developers to customise \Cyclus to analyze the cases that are of interest to them, rather than those the simulator was originally developed to address.  The fundamental concepts of the \Cyclus
nuclear fuel cycle simulator capture these modern insights so that challenges
in nuclear fuel cycle analysis can be better addressed.

\subsection{Background}

% Who else has done what?
% How did they do that?
% What did we do before this?
% What is new?

% Common goals of fuel cycle simulation/simulators (driving &d focus,
% approximating tech. impacts, etc.)
% Previous work (cite all the tools!)
% Limitations of previous simulators by others (fleet-based, closed
% development, inflexible infrastructure)
% Previous attempts by this group (GENIUS v1/v2)
% note NGFCS

Nuclear fuel cycle simulators drive \gls{RDD} by calculating `metrics',
quantitative measures of performance that
can be compared among fuel cycle options. The feasibility of the technology development and deployment
strategies which comprise a fuel cycle option, the dynamics of transitions between fuel cycles, and many other
measures of performance can be expressed in terms of these metrics. For example,
economic feasibility is often measured in levelized cost of electricity (\gls{LCOE}), requiring calculation of
lifetime fuel costs and electricity generation,
while environmental performance might be measured by spent fuel volume,
  toxicity, or mined uranium.  A meta-analysis of fuel cycle systems studies identified over two dozen unique quantitative metrics spanning economics and cost, environmental stainability and waste management impacts, safety, security and nonproliferation, resource adequacy and utilization, among others. With few exceptions, these metrics are derived from mass balances calculated by a fuel cycle simulator. *** shameless plug for some of our work! :)  Citation: Flicker, M. E., Schneider, E. A. and P. Campbell, "Evaluation Criteria for Analyses of Nuclear Fuel Cycles," Trans. Am. Nucl. Soc. 111, November 2014 ***

However, methods for calculating those metrics vary among simulators. Some model the
system of facilities, economics, and materials in static equilibrium, while
other simulators capture the dynamics of the system.
Similarly, while some simulators discretely model batches of material and individual facilities,
others aggregate facilities into fleets and materials into streams. Simulators can *** deleted overspecialization here and below since it struck me as a judgment call rather than objective statement ***
model a single aspect of the fuel cycle in great detail while neglecting
others. For example, a simulator created for policy modeling might have excellent
capability in economics while models affecting transformations in material isotopics or the effects of isotopics on technology performance are
neglected.   *** would be useful to give an example here of a simulator that suffers from overspecification ***

Historically, the national laboratories have driven development and regulated
the use of their own tools: \gls{VISION}\cite{jacobson_verifiable_2010},
\gls{DANESS}\cite{van_den_durpel_daness_2009},
\gls{DYMOND}\cite{yacout_modeling_2005}, and
\gls{NFCSim}\cite{schneider_nfcsim:_2005}.  Outside the national laboratories,
researchers have created new nuclear fuel cycle simulation tools when existing
tools were not available or not sufficiently general to calculate their metrics
of interest.  With limited access to the national
laboratory tools and a need to customize them for research purposes,
universities and private industry researchers have ``reinvented the wheel'' by
developing tools of their own from scratch and tailored to their own needs, for
example \gls{CAFCA}\cite{guerin_impact_2009} and
\gls{COSI}\cite{boucher_cosi_2005,boucher_cosi:_2006,meyer_new_2009,coquelet-pascal_comparison_2011}.
*** I would put COSI into the first category of lab tools, to which you could add ORION.  Nothing else springs to mind here, but I'm sure a quick lit review would reveal some university developed models that were used for a few specific tasks, then forgotten ***

\Cyclus emerged from a line of tools seeking to break this practice.
Its precursor,
\gls{GENIUSv1}\cite{dunzik-gougar_global_2007,jain_transitioning_2006}, originated within \gls{INL} and sought
\gls{ORION}\cite{}, *** lost me here... sought ORION? Also the next sentence seems abrupt... which lessons learned? ***  and
to provide generic regional capability.  Based on lessons learned from
Version 1, the
\acrshort{GENIUSv2} Version 2\cite{oliver_studying_2009,huff_geniusv2_2009} code sought to
provide more generality and an extensible interface to facilitate
collaboration.  The \Cyclus project then improved upon the \acrshort{GENIUSv2}
effort by implementing increased modularity and encapsulation.  The result is
a dynamic simulator that treats both materials and facilities discretely, with
an architecture that permits multiple and variable levels of
fidelity. The simulator's agent-based framework for tracking the
transformation and trade of resources between autonomous entities with customizable behavior and objectives is an innovation not pursued by any existing fuel cycle simulator.   *** changed 'greatly improves upon' because it's subjective, also removed mention of institutions, regions here as the meaning of those terms in Cyclus has not yet been defined ***

\subsection{Motivation}
% Gently reiterate the above need.
% There should be a hero narrative here
% There was a gap in the capabilities of simulators.
% \Cyclus, a hero, came in to fill that gap.
*** I'm inspired to crank up the Bonnie Tyler :)  'I need a hero!' ***

The \Cyclus paradigm enables targeted contribution and collaboration within the nuclear fuel cycle analysis community. This essential capability, absent in previous simulators where user customization and extensibility were not design objectives, comes about through the \emph{modular and open architecture} of \Cyclus.
Additionally, \Cyclus facilitates direct comparison of modeling methodologies with
\emph{agent interchangeability}.
Finally, \Cyclus is applicable to a broader range of
fidelities, scales, and applications than other simulators.
due to the flexibility and generality
of its \emph{\gls{ABM}} paradigm and \emph{discrete, object-oriented approach}

This structure recognizes that specialists should utilize their time and resources in
modeling the specific process associated with their area of expertise (e.g., reprocessing and advanced
fuel fabrication), without having to create a model of the entire fuel cycle to serve as its host.
\Cyclus supports them by separating the problem of modeling a
physics-dependent supply chain into two distinct components: a simulation kernel and archetypes that
interact with the kernel. The kernel is responsible for the
deployment and interaction logic of entities in the simulation.  Physics calculations and
customized behaviors are implemented within archetypes.

Furthermore, modeling the evolution of a physics-dependent, international nuclear fuel supply
chain is a multiscale problem which existing tools cannot support. They have either focused on macro effects, e.g., the fleet-level
stocks and flows of commodities, or micro effects, e.g., the used-fuel
composition of fast reactor fuel. Each focus has driven the
development of specialized tools, rendering the task of answering questions between
the macro and micro levels challenging within a single tool.
In contrast, the extensible architecture and discrete object tracking of \Cyclus allow
the creation and interchangeability of custom archetypes at any level of fidelity.
