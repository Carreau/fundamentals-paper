% Introduction :

\section{Introduction}

% Why did we do the work?
% What were the central motivations and hypotheses?
% The objectives of the work?
% What will this work show the reader?
% Why is it important?
% What should the reader watch for in the paper?
% What are the interesting high points?
% What strategy did we use?
% What should the reader expect as conclusion?


As nuclear power expands both domestically and internationally, nuclear fuel cycle
performance analysis on technical, economic, political, and environmental axes increases 
in importance. Appropriate nuclear fuel cycle analysis requires calculation of myriad physical, nuclear, 
chemical, industrial, and political phenomena. Nuclear fuel cycle simulators 
therefore couple complex systems such as nuclear process physics, 
facility deployment, and material routing. However, most nuclear fuel cycle 
simulators have not incorporated modern insights in simulation science and 
software architecture that can enable more efficient, accurate, robust, and 
validated analysis. 

The \Cyclus Nuclear Fuel Cycle Simulator framework and 
its modeling ecosystem answer this need for a next generation nuclear fuel cycle 
simulator. Dynamic simulation of nuclear fuel cycles with discrete agents 
and up to isotopic resolution support 
levels of detail appropriate for a varied community of analysts
interested in a range of analysis types. Additionally \Cyclus employs an 
open development process that supports user access and 
encourages developer extensions.

Historically, nuclear fuel cycle simulators have estimated 
nuclear fuel cycle performance using software 
ranging from ad hoc spreadsheet-driven flow calculators to highly specialized 
proprietary models. To date, current tools are either static, fleet-based, 
privately distributed, or all of the above. Each of these qualities present a 
severe challenge to modeling fidelity, generality, effciiency, robustness, and 
scientific transparency. 

An alternative to these approaches is a dynamic, agent-based simulator with an 
open platform and an ecosystem of research-driven capabilities.  Modern 
insight indicates that dynamic nuclear fuel cycle analysis more realistically 
supports a broader range of simulation goals than static analysis 
\cite{piet_dynamic_2011}. Similarly, experience in the greater field of systems 
analysis indicates that agent-based modeling enables more flexible simulation 
control, without loss of generality \cite{thatpapermattsent}. Finally, openness 
allows cross-institutional collaboration, increases code robustness, and 
enables the cultivation of an ecosystem of calculation options, 
\cite{softwarecarpentryresource}.  The fundamental concepts of the \Cyclus 
nuclear fuel cycle simulator capture these modern insights so that challenges 
in nuclear fuel cycle analysis can be better addressed. 

\subsection{Background}

% Who else has done what?
% How did they do that?
% What did we do before this?
% What is new?

% Common goals of fuel cycle simulation/simulators (driving r&d focus,
% approximating tech. impacts, etc.)
% Previous work (cite all the tools!)
% Limitations of previous simulators by others (fleet-based, closed
% development, inflexible infrastructure)
% Previous attempts by this group (GENIUS v1/v2)
% note NGFCS

Nuclear fuel cycle simulators can drive \gls{RDD} by calculating `metrics', 
quantitative measures of performance that 
can be compared among fuel cycle options. The feasibility of technology 
strategies, dynamics of transitions between fuel cycles, and many other 
measures of performance can be expressed in terms of these metrics. For example, 
economic feasibility is often measured in \gls{LCOE} while proliferation risk 
can be proportional to total kilograms of separated Pu, and environmental 
performance might be measured in spent fuel volume, toxicity, or mined uranium.

Historically, researchers create new nuclear fuel cycle simulation tools when 
existing tools are not available or not sufficiently general to calculate their 
metrics of interest with their method of choice.

Historically, the national laboratories have driven development and regulated 
the use of their own tools such as \gls{VISION}\cite{jacobson_verifiable_2010}, 
\gls{DANESS}\cite{van_den_durpel_daness_2009}, 
\gls{DYMOND}\cite{modeling_yacout_2005}, and 
\gls{NFCSim}\cite{schneider_nfcsim:_2005}. With limited access to those tools 
and a need to customize them for research purposes, 
universities and private industry researchers ``reinvent the wheel'' by 
developing tools of their own from scratch and tailored to their own needs (e.g., 
\gls{CAFCA}\cite{guerin_benchmark_2006} and \gls{COSI}\cite{cosi} 
respectively). 

This cycle has resulted in a a proliferation of specialized, limited-access 
fuel cycle simulators for the calculation of fuel cycle metrics, and methods 
for calculating those metrics vary among simulators. Some model the 
system of facilities, economics, and materials at static equillibrium, while 
others capture its dynamics.  
Similarly, while some simulators discretely model material and facilities, 
others treat materials in streams and facilities in fleets, in an aggregated 
fashion. Depending on their intended use, simulators often overspecialize, 
modeling a single aspect of the fuel cycle in great detail while neglecting 
others. A simulator created for policy modeling might have excellent 
capability in economics while material isotopics and transmutation detail are 
neglected.

When the \Cyclus simulator arose out of the 
\gls{GENIUSv2}\cite{oliver_studying_2009,huff_geniusv2_2009} code
was the result of lessons-learned from an \gls{LDRD} project at 
\gls{INL}, 
\gls{GENIUSv1}\cite{dunzik-gougar_global_2007,jain_transitioning_2006} which 
captured regional behaviors in fuel cycle scenarios. 

\Cyclus is a dynamic simulator that treats both materials and facilities discretely, with an 
architecture that permits simulations with multiple and variable levels of 
fidelity. Its agent-based framework for simulation and tracking of 
transformation and trade of resources between facilities within institutions 
and regions greatly improves upon the existing field of fuel cycle simulators.

Associated with \Cyclus is Cycamore (the additional modules repository). 

\subsection{Motivation}
% Gently reiterate the above need. 
% There should be a hero narrative here
% There was a gap in the capabilities of simulators.
% \Cyclus, a hero, came in to fill that gap. 

In addition to providing the fundamental features expected in fuel cycle 
simulators, the \Cyclus paradigm enables major capabilites that previous 
simulators lack. First, its \emph{modular and open architecture}  facilitates 
targeted contribution and collaboration within the nuclear fuel cycle analysis 
community.  Additionally, \emph{agent interchangability} facilitates direct 
comparison of modeling methodologies. Finally, the flexibility and generality 
of its \emph{\gls{ABM}} paradigm and \emph{discrete, object-oriented approach} 
emphasizes applicability of the \Cyclus tool accross a broader range of 
applications than other simulators. 

Modeling the evolution of a physics-dependent, international nuclear fuel supply
chain is a difficult problem. To date, it has been treated relatively
lightly. Tools have either focused on macro effects, e.g., the fleet-level
stocks and flows of commodities, or micro effects, e.g., the used fuel
composition fast reactor fuel. Accordingly, each focus has driven the
development of these tools, rendering the task of answering questions between
the macro and micro levels either impossible or difficult and time consuming. 

\Cyclus separates the problem of modeling a
physics-dependent supply chain into a simulation kernel and archetypes that
interact with the kernel during a simulation. The kernel is responsible for the
deployment and interaction logic of entities in the simulation. Physics
calculations are constrained to individual archetypes. Accordingly, specialists
in a given field can utilize their time and resources in modeling a specific
process associated with the nuclear fuel cycle (e.g., reprocessing and advanced
fuel fabrication), rather than also modeling the entirety of the cycle.


\subsubsection{Modular Architecture}
% Motivation for encapsulated plug-in architecture 
% enables ecosystem, ease of contribution across institutions
% enables myriad levels of simulation fidelity
% enables FC analyses that can compare ‘apples to apples'

The encapsulated plug-in architecture within the \Cyclus tool allows developers 
to define nuclear fuel cycle facility processes independent of the simulation 
logic. This design choice acheives two things. First, analysts across 
institutions can take advantage of the simulation logic \gls{API} and 
archetype ecosystem when they apply \Cyclus to their specific problem. The effort they 
undertake can thereby be focused on creating or customizing 
nuclear facility, institution, resource, and toolkit models within their 
specific area of expertise.

% we could use a specific example of Cosi's equivalence method below
Second, because \Cyclus uses a modular archetype approach, comparing two 
archetypes is straightforward. Consider a case in which an analyst would like 
to compare the effect of using different models to determining input fuel 
composition for fast reactors. A variety fuel fabrication archetypes can be 
developed and used interchangably to analyze the effect of each on a 
simulation's output.

We expect that there are a number of types of users and developers who will 
interact with the \Cyclus framework. Those people fall into a number of 
categories. All such people benefit from the modular architecture. 

\subsubsection{Agent-based Paradigm}
% superior detail in capturing simulation dynamics
% more flexible control over behavior
% describe Region/Institution/Facility hierarchy
% note importance of generic resource exchange paradigm

\Cyclus utilizes an \gls{ABM} paradigm.  \gls{ABM} is modern, sophisticated, 
true to reality, and it takes advantage of modern computing resources. 
Furthermore, agent-based modeling provides an analyst a more natural 
perspective from which to model a given problem. In the nuclear fuel cycle 
context for example, an analyst can design an agent with reactor-related logic 
and a separate agent with fuel fabrication-related logic.  

System dynamics popular approach for modeling nuclear fuel cycles in current
tools \cite{VISION,CAFCA}. It has been shown in the literature
\cite{macal_agent-based_2010} that any system dynamics model can be translated
into an agent-based model. Furthermore, it has been shown that agent-based
techniques can provide a richer model than can system dynamics. Formally, the
set of system dynamics models is a strict subset of agent-based models.



\subsubsection{Discrete Objects}
% enables more realistic models and metrics
% material routing metrics
% shadow fuel cycles
% etc.

\Cyclus models facilities, institutions, regions, and resources as discrete 
objects. A discrete resource model allows for a range of modeling granularity. In the
macroscopic extreme, it is equivalent to time-stepped continuous flow. In the
microscopic extreme, the model is capable of representing arbitrarily small 
material objects at isotopic resolution. In this way, the \Cyclus simulator is 
applicable across the full range of modeling extremes. 

This generality specifically permits the use cases that cannot be acheived with 
fleet-based models. That is, some questions require resolution at the level of 
individual facilities and individual materials.  Since fleet-based, 
lumped-material models do not distinguish between discrete facility entities or 
discrete materials, many detailed performance metrics cannot be captured with 
those tools. 

One example use case seeks to capture system vulnerability to nefarious 
material diversion. This type of analysis requires discrete simulation of a 
target facility and the individual materials modified within it and is only 
acheivable with discretely-modeled objects like those in the \Cyclus simulator. 

