% Introduction :

\section{Introduction}

% Why did we do the work?
% What were the central motivations and hypotheses?
% The objectives of the work?
% What will this work show the reader?
% Why is it important?
% What should the reader watch for in the paper?
% What are the interesting high points?
% What strategy did we use?
% What should the reader expect as conclusion?


As nuclear power expands both domestically and internationally, nuclear fuel
cycle performance analysis on technical, economic, political, and environmental
axes increases in importance. Appropriate nuclear fuel cycle analysis requires
calculation of myriad physical, nuclear, chemical, industrial, and political
phenomena. Nuclear fuel cycle simulators therefore couple complex systems such
as nuclear process physics, facility deployment, and material routing.
However, current fuel cycle simulators rely on closed platforms and inflexible
architectures which exhibit three main failure modes. First, they discourage
targeted contribution and collaboration among experts. Next, they hobble
efforts to directly compare modeling methodologies. Finally, they 
over-specialize, rendering most tools applicable to only a subset of
desired simulation fidelities, scales, and applications.
The \Cyclus nuclear fuel cycle simulator framework and 
its modeling ecosystem incorporate modern insights from
simulation science and software architecture to solve these problems. 
As a bonus, many of these methods simultaneously enable more efficient, accurate,
robust, and validated analysis. 

The \Cyclus next-generation fuel cycle simulator is the result of design 
choices made explicitly to:
\begin{itemize}
\item support user access and encourages developer extensions,
\item enable plug-and-play comparison of modeling methodologies,
\item and address a range of analysis types, levels of detail, and analyst sophistication.
\end{itemize}

Historically, nuclear fuel cycle simulators have estimated 
nuclear fuel cycle performance using software 
ranging from ad hoc spreadsheet-driven flow calculators to highly specialized 
system dynamics
proprietary models. To date, current tools are either static, fleet-based, 
privately distributed, or all of the above. Each of these qualities presents a 
severe challenge to modeling fidelity, generality, efficiency, robustness, and 
scientific transparency. 
An alternative to these approaches is a dynamic, agent-based simulator with an 
open platform and an ecosystem of research-driven capabilities.  

\Cyclus, therefore, is a dynamic, agent-based model, which employs a modular 
architecture, an open development process, discrete agents, discrete time, and 
arbitrarily detailed isotopic resolution of materials.  Dynamic nuclear fuel 
cycle analysis more realistically supports a range of simulation goals than 
static analysis 
\cite{piet_dynamic_2011}. Similarly, experience in the broader field of systems 
analysis indicates that agent-based modeling enables more flexible simulation 
control, without loss of generality \cite{macal_agent-based_2010}. Finally, openness 
allows cross-institutional collaboration, increases code robustness \cite{cohen_modern_2010}, and 
cultivates an ecosystem of modeling and calculation options.  The fundamental concepts of the \Cyclus 
nuclear fuel cycle simulator capture these modern insights so that challenges 
in nuclear fuel cycle analysis can be better addressed. 

\subsection{Background}

% Who else has done what?
% How did they do that?
% What did we do before this?
% What is new?

% Common goals of fuel cycle simulation/simulators (driving &d focus,
% approximating tech. impacts, etc.)
% Previous work (cite all the tools!)
% Limitations of previous simulators by others (fleet-based, closed
% development, inflexible infrastructure)
% Previous attempts by this group (GENIUS v1/v2)
% note NGFCS

Nuclear fuel cycle simulators drive \gls{RDD} by calculating `metrics', 
quantitative measures of performance that 
can be compared among fuel cycle options. The feasibility of technology 
strategies, dynamics of transitions between fuel cycles, and many other 
measures of performance can be expressed in terms of these metrics. For example, 
economic feasibility is often measured in \gls{LCOE}, requiring calculation of
lifetime fuel costs and electricity generation, 
while environmental performance might be measured in spent fuel volume, 
  toxicity, or mined uranium.

However, methods for calculating those metrics vary among simulators. Some model the 
system of facilities, economics, and materials in static equilibrium, while 
other simulators capture the dynamics of the system.  
Similarly, while some simulators discretely model material and facilities, 
others treat materials as streams and facilities as fleets, in an aggregated 
fashion. Simulators often overspecialize, 
modeling a single aspect of the fuel cycle in great detail while neglecting 
others. For example, a simulator created for policy modeling might have excellent 
capability in economics while material isotopics and transmutation detail are 
neglected.

This overspecialization and fractured community is the result of history. 
Historically, the national laboratories have driven development and regulated 
the use of their own tools: \gls{VISION}\cite{jacobson_verifiable_2010}, 
\gls{DANESS}\cite{van_den_durpel_daness_2009}, 
\gls{DYMOND}\cite{yacout_modeling_2005}, and 
\gls{NFCSim}\cite{schneider_nfcsim:_2005}.  Outside the national laboratories, 
researchers have created new nuclear fuel cycle simulation tools when existing 
tools were not available or not sufficiently general to calculate their metrics 
of interest with their method of choice.  With limited access to the national 
laboratory tools and a need to customize them for research purposes, 
universities and private industry researchers have ``reinvented the wheel'' by 
developing tools of their own from scratch and tailored to their own needs, for 
example \gls{CAFCA}\cite{guerin_impact_2009} and 
\gls{COSI}\cite{boucher_cosi_2005,boucher_cosi:_2006,meyer_new_2009,coquelet-pascal_comparison_2011}. 

\Cyclus emerged from a line of tools seeking to break this practice.  
Its precursor,
\gls{GENIUSv1}\cite{dunzik-gougar_global_2007,jain_transitioning_2006}, began
as an \gls{LDRD} project within \gls{INL} and sought 
\gls{ORION}\cite{}, and 
to provide generic regional capability.  Based on lessons learned from 
Version 1, the 
\acrshort{GENIUSv2} Version 2\cite{oliver_studying_2009,huff_geniusv2_2009} code sought to 
provide more generality and an extensible interface to facilitate 
collaboration.  The \Cyclus project then improved upon the \acrshort{GENIUSv2} 
effort by implementing increased modularity and encapsulation.  The result is  
a dynamic simulator that treats both materials and facilities discretely, with 
an architecture that permits simulations with multiple and variable levels of 
fidelity. The simulator's agent-based framework for tracking the 
transformation and trade of resources between facilities within institutions 
and regions greatly improves upon the existing fuel cycle simulators.

\subsection{Motivation}
% Gently reiterate the above need. 
% There should be a hero narrative here
% There was a gap in the capabilities of simulators.
% \Cyclus, a hero, came in to fill that gap. 

In addition to providing the fundamental features expected in fuel cycle 
simulators, the \Cyclus paradigm enables essential capabilities that previous 
simulators lack. Critically, targeted contribution and collaboration within the nuclear fuel cycle analysis 
community are assisted by the \emph{modular and open architecture} of \Cyclus.
Additionally, \Cyclus facilitates direct comparison of modeling methodologies with 
\emph{agent interchangeability}.
Finally, \Cyclus is applicable to a broader range of 
fidelities, scales, and applications than other simulators. 
due to the flexibility and generality 
of its \emph{\gls{ABM}} paradigm and \emph{discrete, object-oriented approach} 

Specialists in a given field should utilize their time and resources in 
modeling the specific processes associated with their area of expertise within 
thenuclear fuel cycle (e.g., reprocessing and advanced
fuel fabrication), without having to model the entirety of the cycle.
\Cyclus supports them by separating the problem of modeling a
physics-dependent supply chain into two distinct components: a simulation kernel and archetypes that
interact with the kernel. The kernel is responsible for the
deployment and interaction logic of entities in the simulation.  Physics calculations and 
customized behaviors are implemented within archetypes. 

Furthermore, modeling the evolution of a physics-dependent, international nuclear fuel supply
chain is a multiscale problem, and existing tools are not sufficiently 
multiscale. Tools have either focused on macro effects, e.g., the fleet-level
stocks and flows of commodities, or micro effects, e.g., the used-fuel
composition of fast reactor fuel. Each focus has driven the
development of specialized tools, rendering the task of answering questions between
the macro and micro levels challenging within a single tool. 
In contrast, the extensible architecture and discrete object tracking of \Cyclus allow
the creation and interchangeability of custom archetypes at any level of fidelity.
