% Introduction :

\section{Introduction}

% Why did we do the work?
% What were the central motivations and hypotheses?
% The objectives of the work?
% What will this work show the reader?
% Why is it important?
% What should the reader watch for in the paper?
% What are the interesting high points?
% What strategy did we use?
% What should the reader expect as conclusion?


As nuclear power expands, technical, economic, political, and environmental
analyses of nuclear fuel cycles by simulators increase in importance. The
merits of advanced nuclear technologies and fuel cycles are
shaped by myriad physical, nuclear, chemical, industrial, and political
factors. Nuclear fuel cycle simulators must therefore couple complex models of
nuclear process physics, facility deployment, and material routing.

Indeed, the cardinal purpose of a dynamic nuclear fuel cycle simulator is to calculate
the time- and facility-dependent mass flow through all or part the fuel cycle.
Dynamic nuclear fuel cycle analysis more realistically supports a range of
simulation goals than static analysis \cite{piet_dynamic_2011}. Historically,
dynamic nuclear fuel cycle simulators have calculated fuel cycle mass balances
and performance metrics derived from them using software ranging from
spreadsheet-driven flow calculators to highly specialized system dynamics
modeling platforms.

To date, current tools are typically privately distributed rather than open
source, having been developed in industrial contexts.  Additionally, having
often been developed for customized applications, many possess inflexible
architectures. Finally, many model only fleet-level dynamics of facilities and
materials rather than discrete resolution of those individual agents and
objects.  Those software choices exhibit three main failure modes which present
significant challenges to modeling fidelity, generality, efficiency,
robustness, and scientific transparency. First, they discourage targeted
contribution and collaboration among experts. Next, they hobble efforts to
directly compare modeling methodologies. Finally, they over-specialize,
rendering most tools applicable to only a subset of desired simulation
fidelities, scales, and applications.

The \Cyclus nuclear fuel cycle simulator framework and its \emph{modeling
ecosystem}, the suite of agents and other physics plug-in libraries compatible with it,
incorporate modern insights from simulation science and software architecture
to solve these problems.  These modern methods simultaneously enable more
efficient, accurate, robust, and validated analysis.  This next-generation fuel
cycle simulator is the result of design choices made to:

\begin{itemize}
\item support access to the tool by fuel cycle analysts and other users,
\item encourage developer extensions,
\item enable plug-and-play comparison of modeling methodologies,
\item and address a range of analysis types, levels of detail, and analyst sophistication.
\end{itemize}

\Cyclus is a dynamic, agent-based model, which employs a modular architecture,
an open development process, discrete agents, discrete time, and arbitrarily
detailed isotopic resolution of materials. Experience in the broader field of
systems analysis indicates that agent-based modeling enables more flexible
simulation control, without loss of generality
\cite{macal_agent-based_2010}. Furthermore, openness allows cross-institutional
collaboration, increases software robustness \cite{cohen_modern_2010}, and
cultivates an ecosystem of modeling options. This ecosystem is \emph{modular},
being comprised of dynamically loadable, interchangeable, plug-in libraries of fuel cycle component
process physics that vary in their scope, depth, and fidelity. This modularity
allows users and developers to customise \Cyclus to analyze the cases that are
of interest to them, rather than any custom application the simulator was
originally developed to address.  The fundamental concepts of the \Cyclus
nuclear fuel cycle simulator capture these modern insights so that novel
challenges in nuclear fuel cycle analysis can be better addressed.

\subsection{Background}

% Who else has done what?
% How did they do that?
% What did we do before this?
% What is new?

% Common goals of fuel cycle simulation/simulators (driving &d focus,
% approximating tech. impacts, etc.)
% Previous work (cite all the tools!)
% Limitations of previous simulators by others (fleet-based, closed
% development, inflexible infrastructure)
% Previous attempts by this group (GENIUS v1/v2)
% note NGFCS

Nuclear fuel cycle simulators drive \gls{RDD} by calculating `metrics',
quantitative measures of performance that can be compared among fuel cycle
options.  The feasibility of the technology development and deployment
strategies which comprise a fuel cycle option, the operational features of
nuclear energy systems, the dynamics of transitions between fuel cycles, and
many other measures of performance can be expressed in terms of these metrics.
For example, economic feasibility is often measured in \gls{LCOE}, a
combination of fuel and operating costs normalized by electricity generation, while
environmental performance might be measured by spent fuel volume, toxicity, or
mined uranium.  A meta-analysis of fuel cycle systems studies identified over
two dozen unique quantitative metrics spanning economics and cost,
environmental sustainability and waste management impacts, safety, security and
nonproliferation, resource adequacy and utilization, among others
\cite{flicker_evaluation_2014}. With few exceptions, these metrics are derived from
mass balances and facility operation histories calculated by a fuel cycle
simulator. For example, where nuclear waste repository burden is derived from
ejected fuel masses, water pollution or land use can be derived from facility
operational histories (as in \cite{poinssot_assessment_2014}).

However, methods for calculating those metrics vary among simulators. Some
model the system of facilities, economics, and materials in static equilibrium,
while other simulators capture the dynamics of the system.  Similarly, while
some simulators discretely model batches of material and individual facilities,
others aggregate facilities into fleets and materials into streams. Some
simulators were designed to model a single aspect of the fuel cycle in great
detail while neglecting others. For example, a simulator created for policy
modeling might have excellent capability in economics while capabilities for
tracking transformations in material isotopics and the effects of isotopics on
technology performance are neglected.  The \gls{CAFCA}\cite{guerin_impact_2009}
simulator is problem-oriented in this way, having elected to neglect isotopic
resolution in favor of integral effects.

Historically, domestic national laboratories have driven development and
regulated the use of their own tools:
\gls{VISION}\cite{jacobson_verifiable_2010},
\gls{DYMOND}\cite{yacout_modeling_2005}, and
\gls{NFCSim}\cite{schneider_nfcsim:_2005,allan_guidance_2008}.  Internationally,
other laboratories have created their own as well, such as
\gls{COSI}\cite{boucher_cosi_2005,boucher_cosi:_2006,meyer_new_2009,coquelet-pascal_comparison_2011}
and ORION\cite{worrall_scenario_2007}.  Finally, some simulators initiated in a national lab setting have
been continued as propriety, industry-based simulators, such as
\gls{DANESS}\cite{van_den_durpel_daness_2009}.  Outside the national laboratories,
researchers have created new nuclear fuel cycle simulation tools when existing
tools were not available or not sufficiently general to calculate their metrics
of interest.  With limited access to the
national laboratory tools and a need to customize them for research purposes,
universities and private industry researchers have ``reinvented the wheel'' by
developing tools of their own from scratch and tailored to their own needs.
Examples include \gls{CAFCA}\cite{guerin_benchmark_2009} and
\gls{DESAE}\cite{andrianova_desae_2008,mccarthy_benchmark_2012,allan_guidance_2008}.

\Cyclus emerged from a line of tools seeking to break this practice.  Its
precursor, \gls{GENIUS} Version 1
\cite{dunzik-gougar_global_2007,jain_transitioning_2006}, originated within
\gls{INL} and sought to provide generic regional capability.  Based on lessons
learned from \gls{GENIUS} Version 1, the \gls{GENIUS} Version 2
\cite{oliver_studying_2009,huff_geniusv2_2009} simulator sought to provide more
generality and an extensible interface to facilitate collaboration.  The \Cyclus
project then improved upon the \gls{GENIUS} effort by implementing increased
modularity and encapsulation.  The result is a dynamic simulator that treats
both materials and facilities discretely, with an architecture that permits
multiple and variable levels of fidelity. Using an agent-based framework, the
simulator tracks the transformation and trade of resources between autonomous
regional and institutional entities with customizable behavior and
objectives. This capability is an innovation not pursued by any existing fuel
cycle simulator.


\subsection{Motivation}
% Gently reiterate the above need.
% There should be a hero narrative here
% There was a gap in the capabilities of simulators.
% \Cyclus, a hero, came in to fill that gap.

The \Cyclus paradigm enables targeted contribution and collaboration within the
nuclear fuel cycle analysis community to achieve two important goals: lower the
barrier for users to include custom nuclear technologies in their fuel cycle
analyses while improving the ability to compare simulations with and without
those custom concepts.  This essential capability is absent in
previous simulators where user customization and extensibility were not design
objectives.  While the \emph{modular and open architecture} of
\Cyclus is necessary to meet these goals, it is not sufficient.
\emph{Agent interchangeability} is also required to facilitates direct comparison
of alternative modeling methodologies and facility concepts. With this concept
at its core, \Cyclus provides a platform for users to quickly develop the
capabilities at a level of detail and validation necessary for their
unique applications.  Finally, \Cyclus is applicable to a broader range of
fidelities, scales, and applications than other simulators, due to the
flexibility and generality of its \emph{\gls{ABM}} paradigm and \emph{discrete,
object-oriented approach}.

This structure recognizes that specialists should utilize their time and
resources in modeling the specific process associated with their area of
expertise (e.g., reprocessing and advanced fuel fabrication), without having to
create a model of the entire fuel cycle to serve as its host.  \Cyclus supports
them by separating the problem of modeling a physics-dependent supply chain into
two distinct components: a simulation kernel and archetypes that interact with
it. The kernel is responsible for supporting the deployment and
interaction logic of entities in the simulation.  Physics calculations and
customized behaviors of those entities are implemented within \emph{archetype}
classes.

Ultimately, modeling the evolution of a physics-dependent, international
nuclear fuel supply chain is a multi-scale problem which existing tools cannot
support. They have either focused on macro effects, e.g., the fleet-level
stocks and flows of commodities, or micro effects, e.g., the used-fuel
composition of fast reactor fuel. Each focus has driven the development of
specialized tools, rendering the task of answering questions between the macro
and micro levels challenging within a single tool.  In contrast, the open, extensible
architecture and discrete object tracking of \Cyclus allow the creation and
interchangeability of custom archetypes at any level of fidelity and by any
fuel cycle analyst.


\subsubsection{Open Access and Development Practices}

The proprietary concerns of research institutions and security constraints of
data within fuel cycle simulators often restrict access. Use of a simulator is
therefore often limited to its institution of origin, necessitating effort
duplication at other institutions and thereby squandering broader human
resources. License agreements and institutional approval are required for most
current simulators (e.g. \gls{COSI}6, \gls{DANESS}, \gls{DESAE}, EVOLCODE,
FAMILY21, \gls{NFCSim})\cite{juchau_modeling_2010}, including ORION,
and \gls{VISION}.  Even when the source code is unrestricted, the platform on which it relies (e.g. VENSIM) 
is often restricted or costly. The MIT \gls{CAFCA} software, for example, 
relies on the commercially licensed VENSIM product as a platform.
\Cyclus, on the other hand, is written in C++ for which freely available 
developement tools and an open standard are available. Further, \Cyclus relies 
only on open source, freely available libraries. As such, it provides fully 
free and open access to all users and developers, foreign and domestic.

Moreover, both technical and institutional aspects of the software development
practices employed by the \Cyclus community facilitate collaboration.
Technically, \Cyclus employs a set of tools commonly used collaborative
software development that reduce the effort required to comment on, test and
ultimately merge individual contributions into the main development path.
For many of the simulation platforms adopted by previous simulators, there were
technical obstacles that impeded this kind of collaboration.
Institutionally, \Cyclus invites all participants to propose, discuss and
provide input to the final decision making for all important changes.

\subsubsection{Modularity and Extensibility}

Modularity is a key enabler of extending the scope of fuel cycle analysis
within the \Cyclus framework.  Changes that are required to improve the
fidelity of modeling a particular agent, or to introduce entirely new agents,
are narrowly confined and place no new requirements on the \Cyclus kernel.
Furthermore, there are few assumptions or heuristics that would otherwise
restrict the algorithmic complexity that can be used to model the behavior of
such agents.

For example, most current simulators describe a finite set of acceptable cycle
constructions (once through, single-pass, multi-pass). That limits the
capability to create novel material flows and economic scenarios. The \Cyclus
simulation logic relies on a market paradigm, parameterized by the user, which
flexibly simulates dynamic responses to pricing, availability, and other
institutional preferences.

This minimal set of mutual dependencies between the kernel and the agents is
expressed through the \gls{DRE} that provides a level of flexibility that does
not exist in other fuel cycle simulators.  It creates the potential for novel
agent archetypes to interact with existing archetypes as they enter and leave
the simulation over time and seek to trade materials whose specific
composition may not be known \textit{a priori}.

\subsubsection{Discrete Facilities and Materials}

Many fuel cycle phenomena have aggregate system-level effects which can only be
captured by discrete material tracking \cite{huff_next_2010}.  \Cyclus
tracks materials as discrete objects. Some current fuel cycle simulation tools
such as \gls{COSI}
\cite{mccarthy_benchmark_2012,grasso_nea-wpfc/fcts_2009,guerin_benchmark_2009},
FAMILY21\cite{mccarthy_benchmark_2012},
\gls{GENIUS} version 1, \gls{GENIUS} version 2, \gls{NFCSim}, and ORION also
possess the ability to model discrete materials. However, even among these, the ability to model reactor facilities individually is not equivalent to the ability to model distinct activities. \gls{COSI}, for example,
has some support for modeling reactors individually, but according to a recent
benchmark \cite{boucher_benchmark_2012}, it models many reactors operating in sync. That is, refuelling and discharging occur simultaneously for all reactors.
While \Cyclus allows this type of fleet-based aggregation of reactor behavior, \Cyclus also enables operations in each facility to vary independently of any others in the simulation.

Similarly, the ability to model disruptions (i.e. facility shutdowns due to
insufficient feed material or insufficient processing and storage capacity) is
most readily captured by software capable of tracking the operations status of
discrete facilities \cite{huff_next_2010}.  Fleet-based models (e.g.
\gls{VISION}) are unable to capture this gracefully, since supply disruptions
are modeled as a reduction in the capacity of the whole fleet.  All of the
software capable of discrete materials have a notion of discrete facilities,
however not all handle disruption in the same manner. \gls{DESAE}, for example,
does not allow shutdown due to insufficient feedstock. In the event of
insufficient fissile material during reprocessing, \gls{DESAE} borrows material
from storage, leaving a negative value \cite{mccarthy_benchmark_2012}.  The
\Cyclus framework does not dictate such heuristics. Rather, it provides a
flexible framework on which either method is possible.

A final benefit of the discreteness of facilities and materials is their power
when combined. The ability to track a material's history as it moves from one
facility to another is unique to \Cyclus. While some current simulators track
materials in discrete quanta, they do not necessarily preserve the identity of each quantum as the
materials move around the fuel cycle. When coupled with \Cyclus' individual
facility modeling, this capacity becomes distinct from what other fuel cycle
simulators are able to do. So, while FAMILY21 and \gls{COSI} can identify
whether a batch being discharged from a reactor originated in \gls{MOX}
fabrication rather than fresh \gls{UOX} fabrication, \Cyclus can go further,
tracking which of the fresh batches contained material from a particular
discharged batch. By extension, \Cyclus can also report which individual
facilities the batch passed through and in which it originated. 
The ability to track a single material through the simulation, 
though it might be split, transmuted, or merged with other materials along the 
way, allows \Cyclus to answer more data-rich questions that previous simulators
have been unable to ask. For example, it allows precise tracking of 
specific material diversions, so queries about nonproliferation 
robustness in a facility can be levied either in the context of a single event or 
a series of nefarious acts.


