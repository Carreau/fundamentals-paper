% Introduction :

\section{Introduction}

% Why did we do the work?
% What were the central motivations and hypotheses?
% The objectives of the work?
% What will this work show the reader?
% Why is it important?
% What should the reader watch for in the paper?
% What are the interesting high points?
% What strategy did we use?
% What should the reader expect as conclusion?

In nuclear fuel cycle simulation, it is necessary to calculate and optimize 
nuclear process physics, facility deployment, and material routing in order to 
capture overall fuel cycle performance metrics. 

Historically, nuclear fuel cycle simulators have addressed this with software 
ranging from spreadsheet driven flow calculators to very high-fidelity industrial 
models. 

An alternative to these approaches is a dynamic, agent-based simulator with an 
open platform.

\subsection{Background}

% Who else has done what?
% How did they do that?
% What did we do before this?
% What is new?

% Common goals of fuel cycle simulation/simulators (driving r&d focus,
% approximating tech. impacts, etc.)
% Previous work (cite all the tools!)
% Limitations of previous simulators by others (fleet-based, closed
% development, inflexible infrastructure)
% Previous attempts by this group (GENIUS v1/v2)
% note NGFCS

Simulators drive R\&D by calculating metrics that can be compared accross fuel 
cycles. 

Previous simulators have been static or dynamic. Some have been discrete while 
others have been fleet-based. Some have captured regional behavior, material 
isotopics, transmutation detail, etc. Indeed, some tools have attempted 
agent-based functionality.

Historically, the national laboratories have driven development and regulated 
the use of their own tools (cite VISION, DANESS, NFCSim, etc.), while 
universities and industry have simultaneously developed tools of their own as 
well. 

The Cyclus code arose out of the GENIUSv2 code. That software itself was the 
result of lessons-learned from an LDRD project at INL called GENIUSv1. 


\subsection{Motivation}
% Gently reiterate the above need. 
% There should be a hero narrative here
% There was a gap in the capabilities of simulators.
% Cyclus, a hero, came in to fill that gap. 

Cyclus provides a sophisticated and modern, agent-based platform. Previous 
codes were lacking in a number of ways. Cyclus has a number of features that 
overcome those issues.  

\subsubsection{Modular Architecture}
% Motivation for encapsulated plug-in architecture 
% enables ecosystem, ease of contribution across institutions
% enables myriad levels of simulation fidelity
% enables FC analyses that can compare ‘apples to apples'

Draw from and cite the ``Open Architecture and Modular Paradigm'' ANS summary.

We expect that there are a number of types of users and developers who will 
interact with the Cyclus framework. Those people fall into a number of 
categories. All such people benefit from the modular architecture. 

<modified-open-source collaboration diagram>

\subsubsection{Agent-based Paradigm}
% superior detail in capturing simulation dynamics
% more flexible control over behavior
% describe Region/Institution/Facility hierarchy
% note importance of generic resource exchange paradigm

ABM is modern, sophisticated, true to reality, and it takes advantage of modern 
computing resources. 


\subsubsection{Discrete Resource Tracking}
% enables more realistic models and metrics
% material routing metrics
% shadow fuel cycles
% etc.

Not all simulations can be successfully run with a fleet-based model that 
doesn't distinguish between discrete materials. Shadow fuel cycles are an 
example. 

