% Introduction :

\section{Introduction}

% Why did we do the work?
% What were the central motivations and hypotheses?
% The objectives of the work.
% What will this work show
% Why is it important?
% What should the reader watch for in the paper?
% What are the interesting high points?
% What strategy did we use?
% What should the reader expect as conclusion?



\subsection{Background}

% Who else has done what?
% How did they do that?
% What did we do before this?
% What is new?

% Common goals of fuel cycle simulation/simulators (driving r&d focus,
% approximating tech. impacts, etc.)
% Previous work (cite all the tools!)
% Limitations of previous simulators by others (fleet-based, closed
% development, inflexible infrastructure)
% Previous attempts by this group (GENIUS v1/v2)
% note NGFCS

\subsection{Motivation}
% Gently reiterate the above need. 
% There should be a hero narrative here
% There was a gap in the capabilities of simulators.
% Cyclus, a hero, came in to fill that gap. 

\subsubsection{Modular Architecture}
% Motivation for encapsulated plug-in architecture 
% enables ecosystem, ease of contribution across institutions
% enables myriad levels of simulation fidelity
% enables FC analyses that can compare ‘apples to apples'
\subsubsection{Agent-based Paradigm}
% superior detail in capturing simulation dynamics
% more flexible control over behavior
% describe Region/Institution/Facility hierarchy
% note importance of generic resource exchange paradigm
\subsubsection{Discrete Resource Tracking}
% enables more realistic models and metrics
% material routing metrics
% shadow fuel cycles
% etc.

